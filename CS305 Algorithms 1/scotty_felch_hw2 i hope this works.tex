\documentclass{article}
\usepackage{geometry,amsmath}
\usepackage[american]{babel}
\geometry{letterpaper}

%%%%%%%%%% Start TeXmacs macros
\newcommand{\tmem}[1]{{\em #1\/}}
\newcommand{\tmop}[1]{\ensuremath{\operatorname{#1}}}
\newcommand{\tmstrong}[1]{\textbf{#1}}
%%%%%%%%%% End TeXmacs macros

\begin{document}

Scotty Felch

CSCI 305

4 February 2015

HW \#2



1a. When {\tmem{found}} is {\tmstrong{true}}, then the value for x is

\ \ \ \ \ \ \ \ \ A[low] + A[high] == x.

\ \ \ \ This is the only way for the control flow of the code to enter the
while loop beginning on line 5:

\ \ \ \ \ \ \ \ \ \ \ ((low < high) AND NOT found)

\ \ \ \ and the if statement on line 6:

\ \ \ \ \ \ \ \ \ \ if (A[low] + A[high] == x).



1b. For notation purposes, let A be the original array, and S be set of
eliminated possibilities (the two tails \ \ \ \ \ of the array).

\ \ \ \ We can assume that at the start of each iteration, S is still the set
of all eliminated possibilities, and \ \ \ \ \ \ \ \ \ {\tmem{found}} ==
false.

\ \ \ \ Initiliazation: Low = 0, high = n-1, and S = \{null\}.

\ \ \ \ \ \ \ \ \ \ \ Assuming there is a winning combination in the array,
then

\ \ \ \ \ \ \ \ \ \ \ P(win) = P(low wins) * P(high wins).

\ \ \ \ \ \ \ \ \ \ \ \ \ \ \ \ \ \ \ \ \ = (1/n) * (1/n-1) = 1/(n(n-1))

\ \ \ Maintenance: Low >= 1, \ high <= n-2

\ \ \ \ \ \ \ \ \ S = [0 .. low-1] and [high+1 .. n-1], \ \ by initial
assumption.

\ \ \ \ \ \ \ \ \ For every iteration through this algorithm, while
{\tmem{found}} == false, either {\tmem{low}} will increment or {\tmem{high}} \
\ \ \ \ \ \ \ \ \ will decrement. This grows the size of S, while narrowing
down the possibilities remaining in A that \ \ \ \ \ \ \ \ \ could be the
winning match. This process continues for each iteration until either we find
the match \ \ \ \ \ \ \ \ \ and set found == true, or run out of options and
enter the following Termination case.



\ \ \ \ Termination: Low = high + 1, \ \ high = low -1, and S = A.

\ \ \ \ \ \ \ \ \ \ \ P(win) = 0, since at this point we've failed. \ \ \ \



2a. \ \ A = [2, 3, 8, 6, 1].

\ \ \ \ \ \ Inversions = (2,1), (3,1), (8,6), (8,1), (6,1)

2b. \ \ What array with elements from the set \{1, 2, ..., n\} has the most
inversions?

\ \ \ \ \ \ A reverse ordered array.

\ \ \ \ \ \ How many inversions does it have?

\ \ \ \ \ \ $\underset{i = 1}{\overset{}{} \overset{n}{\sum} \overset{}{} } n
- i$ \ \ => \ \ $\frac{n^2 + n}{2}$ \ \ \ \ (by Gaussian identity)



2c. \ \ What is the relationship between running time of insertion sort and
the number of inversions in the \ \ \ \ \ \ \ \ \ \ input array?

\ \ \ \ - $\theta ( n^2)$ is worst case of insertion sort.

\ \ \ \ - Every inversion is an operation that must be performed.

\ \ \ \ - n inversions translates to $\underset{i = 1}{\overset{n}{\sum}} ( n
- 1) $ \ operations that must be performed in addition to overhead costs.

\ \ \ - $\underset{i = 1}{\overset{n}{\sum}} ( n - 1) $ \ simplifies to
$\frac{n^2 + n}{2} .$

\ \ \ - This makes sense because $\frac{n^2 + n}{2}$ translates to $\theta (
n^2)$ when you drop off the low order terms.

2d. \ \ \ An algorithm that determines the number of inversions in any
permutation on n elements in \ \ \ \ \ \ \ \ \ \ \ \ \ \ \ \ \ \ \ \ $\theta (
n \lg n) \tmop{worst} \tmop{case} \tmop{time}$ is a simple modification of the
Merge Sort algorithm on p. 31 of the text. The algorithm is already tracking
every occurrence of an inversion as part of the sorting process, this occurs
when the else block on line 16 is hit. However you can't just add in a
counter, since the sorting process intermediate steps would cause some
inversions to be not counted.

The two changes two make would be to change lines 16-17 to:

\ \ \ \ \ \ \ else A[k] = L[i]

\ \ \ \ \ \ \ \ \ \ \ j = j + 1

\ \ \ \ \ \ \ \ \ \ \ inversion\_counter = inversion\_counter + 1



3a. I used the problem in textbook on page 121 for reference on this, it is
essentially the same format.

\ \ \ \ X$_i $= I\{you guess correctly\}

\ \ \ \ \ \ \ \ = \{1 if card guess i is correct, \ \ \ \ 0 if card guess i
is incorrect

\ \ \ \ \ n = \ total cards

\ \ \ \ X \ = X$_1$ + X$_2$ + ... + X$_n$ \ \ \ \ \ \ \ \ \ \ \ \ \ \ \ (X is
the total of all trials completed)

\ \ \ E[X] = how many guesses correct in total

\ \ \ E[X$_i$] = P(i$^{\tmop{th}}$ guess is correct) = $\frac{1}{n}$

\ \ \ E[X] = E[$\underset{i = 1}{\overset{n}{\sum}}$X$_i$ ] = $\underset{i =
1}{\overset{n}{\sum}}$E[X$_i$] = $\underset{i = 1}{\overset{n}{\sum}}
\frac{1}{n}$ \ \ = $\frac{1}{n}$*n = $\frac{n}{n}$ = 1 correct guess



3b. E[X$_i$] = P(i$^{\tmop{th}}$ guess is correct) = $\frac{1}{n - i + 1}$ \ \
\ \ \ because your \# of cards to draw from decreases each draw

\ \ \ E[X] = E[$\underset{i = 1}{\overset{n}{\sum}}$X$_i$ ] = $\underset{i =
1}{\overset{n}{\sum}}$E[X$_i$] = $\underset{i = 1}{\overset{n}{\sum}}$
$\frac{1}{n - i + 1}$ = $\underset{i = 1}{\overset{n}{\sum}} \frac{1}{i}$ \ =
ln n + O(1)

\end{document}
